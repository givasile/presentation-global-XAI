%%%%%%%%%%%%%%%%%%%%%%%%%%%%%%%%%%%%%%%%%%%%%%%%%%%%%%%%%%%%%%%
%
% Welcome to Overleaf --- just edit your LaTeX on the left,
% and we'll compile it for you on the right. If you open the
% 'Share' menu, you can invite other users to edit at the same
% time. See www.overleaf.com/learn for more info. Enjoy!
%
%%%%%%%%%%%%%%%%%%%%%%%%%%%%%%%%%%%%%%%%%%%%%%%%%%%%%%%%%%%%%%%
\documentclass{beamer}
\usepackage{tikz}
\usepackage{amsmath, bm, amssymb, amsfonts}
\usepackage{graphicx}
\graphicspath{{figures/chapter1} {figures/chapter2} {figures/chapter3} {figures/chapter4}}
\usepackage{xcolor}
\usepackage{hyperref}
\usepackage{algorithm}
\usepackage[noend]{algpseudocode}
\hypersetup{
    colorlinks=true,
    linkcolor=blue,
    filecolor=magenta,
    urlcolor=cyan,
    }
\usepackage[backend=biber, citestyle=authoryear, bibstyle=alphabetic]{biblatex}
\addbibresource{bibliography.bib}

\newcommand{\Vu}{\mathbf{u}}
\newcommand{\Vh}{\mathbf{h}}
\newcommand{\Vv}{\mathbf{v}}
\newcommand{\Vw}{\mathbf{w}}
\newcommand{\Vb}{\mathbf{b}}
\newcommand{\Vx}{\mathbf{x}}
\newcommand{\Vy}{\mathbf{y}}
\newcommand{\Vz}{\mathbf{z}}
\newcommand{\Vf}{\mathbf{f}}
\newcommand{\VC}{\mathbf{C}}
\newcommand{\VD}{\mathbf{D}}
\newcommand{\VX}{\mathbf{X}}
\newcommand{\VS}{\mathbf{S}}
\newcommand{\VW}{\mathbf{W}}
\newcommand{\VV}{\mathbf{V}}
\newcommand{\VU}{\mathbf{U}}
\newcommand{\Vth}{\bm{\theta}}
\newcommand{\pmodel}{p_{\text{model}}}
\newcommand{\pdata}{\hat{p}_{\text{data}}}
\newcommand{\thetab}{\boldsymbol{\theta}}
\newcommand{\xb}{\boldsymbol{x}}



\usepackage{tikzsymbols}
\usetheme{Frankfurt}
\usecolortheme{seahorse}
\DeclareMathOperator*{\argmin}{arg\,min}
\DeclareMathOperator*{\argmax}{arg\,max}

\title[FUB-Presentation]{Global Explainability (XAI) Techniques}
\subtitle{State of the Art and Challenges}
\author[Gkolemis, Vasilis] % (optional)
{Vasilis Gkolemis\inst{!}\inst{*}}

\institute[HUA] % (optional)
{
  \inst{!} ATHENA Research and Innovation Center \and
  \inst{*} Harokopio University of Athens
}

\date{March 2023}


% Use a simple TikZ graphic to show where the logo is positioned
% \logo{\begin{tikzpicture}
% \filldraw[color=red!50, fill=red!25, very thick](0,0) circle (0.5);
% \node[draw,color=white] at (0,0) {LOGO HERE};
% \end{tikzpicture}}

%End of title page configuration block
%------------------------------------------------------------
%The next block of commands puts the table of contents at the
%beginning of each section and highlights the current section:

\AtBeginSection[]
{
  \begin{frame}
    \frametitle{Program}
    \tableofcontents[currentsection]
  \end{frame}
}

% ------------------------------------------------------------
\begin{document}
\frame{\titlepage}
%---------------------------------------------------------

% chapter 1
\section{Intro to XAI}
\begin{frame}
  \frametitle{Hypothetical (?) scenarios}

  \begin{itemize}
  \item<1-> The computer vision subsystem of an autonomous vehicle leads the
    vehicle to take a left turn, in front of a car moving in the opposite direction\footnote{\url{https://www.theguardian.com/technology/2022/dec/22/tesla-crash-full-self-driving-mode-san-francisco}}
  \item<2-> The credit assessment system leads to the rejection of an
    application for a loan - the client suspects racial bias\footnote{\url{https://www.technologyreview.com/2021/06/17/1026519/racial-bias-noisy-data-credit-scores-mortgage-loans-fairness-machine-learning/}}
  \item<3-> A model that assesses the risk of future criminal offenses (and
    used for decisions on parole sentences) is biased against black
    prisoners\footnote{\url{https://www.propublica.org/article/machine-bias-risk-assessments-in-criminal-sentencing}}
  \end{itemize}

\end{frame}

\begin{frame}
  \frametitle{Questions}
  \begin{itemize}
  \item Why did the model make a specific decision? \textcolor{red}{local XAI}
  \item What could we change so that the model will make a different decision? \textcolor{red}{counterfactual}
  \item Can we summarize the model's behavior? \textcolor{red}{global XAI}
  \item Models as knowledge extractors, what hat the model learnt \textcolor{red}{global XAI}
  \end{itemize}
\end{frame}

\begin{frame}
  \frametitle{Interpretability of Machine Learning Models}
  Qualitative definitions:
  \begin{itemize}
  \item<1-> ``Interpretability is the degree to which a human can understand the
    cause of a decision'' \footnote{Miller, Tim. ``Explanation in artificial
    intelligence: Insights from the social sciences.'' arXiv Preprint
    arXiv:1706.07269. (2017)}
  \item<2-> ``Interpretability is the degree to which a human can consistently
    predict the model’s result''\footnote{Kim, Been, Rajiv Khanna, and
    Oluwasanmi O. Koyejo. ``Examples are not enough, learn to criticize!
    Criticism for interpretability.'' Advances in Neural Information Processing
    Systems (2016).}
  \item<3-> ``Extraction of relevant knowledge from a machine-learning model
    concerning relationships either contained in data or learned by the
    model''\footnote{Murdoch, W. J., Singh, C., Kumbier, K., Abbasi-Asl, R. and
    Yu, B. ``Definitions, methods, and applications in interpretable machine
    learning.'' Proceedings of the National Academy of Sciences, 116(44),
    22071-22080. (2019)}
  \end{itemize}
\end{frame}

\begin{frame}
  \frametitle{My understanding}
  Interpretability is the degree to which a human can understand the reasoning process for a (specific) prediction

  \begin{itemize}
    \item interpretability: either by-design or assisted by a post-hoc XAI technique
    \item degree: non binary, interpretability is a spectrum
    \item human: interpretability is a human-centric procedure
    \item reasoning process: mechanism for predicting
  \end{itemize}
  \end{frame}

\begin{frame}
  \frametitle{Global vs Local}
  \begin{itemize}
    \item \textcolor{red}{Local}
    \begin{itemize}
      \item Interpret the model's output for a particular input
      \item Extract interpretable quantity that holds for $x$ close to $x^{(i)}$
    \end{itemize}

    \item \textcolor{red}{Global}
    \begin{itemize}
      \item Provide a general interpretation of the model's behavior
      \item Extract interpretable quantity that holds for $x \in \mathcal{X}$
    \end{itemize}
  \end{itemize}

  \begin{figure}
    \centering
    \includegraphics[width=0.37\linewidth]{ale_concept}
    \includegraphics[width=0.4\linewidth]{lime_concept}
    \caption{(Left) Global vs (Right) Local}
  \end{figure}
\end{frame}


\begin{frame}
  \frametitle{Challenges on global methods}
  Extract an interpretable quantity that \textcolor{red}{holds for $x \in \mathcal{X}$}
  \begin{itemize}
    \item Fidelity: does the interpretable quantity mimics the model's behavior?
    \item Interpretability: is the extracted quantity interpretable enough?
    \item Can we have both?
    \begin{itemize}
      \item if yes, why not replacing the original model with an interpretable one?
      \item if no, how to deal with the trade-off?
    \end{itemize}
  \end{itemize}

  \noindent\makebox[\linewidth]{\rule{\paperwidth}{0.4pt}}
  Spoiler: Maybe uncertainty helps...
\end{frame}


\begin{frame}
  \frametitle{Methods we will discuss}

  \begin{itemize}
  \item Feature Effect
    \begin{itemize}
      \item Output: mapping \(f_s(x_s): x_s \rightarrow y\)
      \item Interpretation: isolate the effect of a single feature \(x_s\) on the output \(y\)
      \item Paper: \href{https://arxiv.org/abs/1612.08468}{Visualizing the Effects of Predictor Variables in Black Box Supervised Learning Models}
    \end{itemize}
  \item Feature Interaction
    \begin{itemize}
    \item Output: a number, the level of interaction between \(x_i\) and \(x_j\)
    \item Interpretation: to what extent the prediction can be expressed as the sum of the feature effects
      \item Paper: \href{https://arxiv.org/abs/1805.04755}{A Simple and Effective Model-Based Variable Importance Measure}
    \end{itemize}
  \item Feature Importance
    \begin{itemize}
      \item Output: a number that expresses the importance of feature \(x_s\) on the output
      \item Interpretation: to what extent the prediction accuracy would drop, if \(x_s\) was asbsent
        \item Paper: \href{https://arxiv.org/abs/1801.01489}{All Models are Wrong, but Many are Useful}
    \end{itemize}

  \end{itemize}

\end{frame}


% chapter 2
\section{Feature Effect}
%\subsection{Feature Effect}

\begin{frame}
  \frametitle{Example}
  \begin{onlyenv}<1>
    Consider the following mapping $x \rightarrow y$
    \begin{center}
      \scalebox{0.5}{
        \input{figures/chapter2/ovf_1_reality.pgf}
      }
    \end{center}
  \end{onlyenv}
  \begin{onlyenv}<2>
    Process unknown \(\rightarrow\) we only have samples
    \begin{center}
      \scalebox{0.5}{
        \input{figures/chapter2/ovf_2_sampling.pgf}
      }
    \end{center}
  \end{onlyenv}
  \begin{onlyenv}<3>
    Our goal is to model the process using the available samples (regression)
    \vspace{1cm}\\
  \end{onlyenv}
  \begin{onlyenv}<4>
    Linear model \(\rightarrow\) Underfiting!
    \begin{equation*}
      y = w_1\cdot x + w_0
    \end{equation*}
    \begin{center}
      \scalebox{0.5}{
        \input{figures/chapter2/ovf_3_linear.pgf}
      }
    \end{center}
  \end{onlyenv}
  \begin{onlyenv}<5>
    2$^{nd}$ degree polynomial \(\rightarrow\) Decent Fit!
    \begin{equation*}
      y = w_2\cdot x^2 + w_1\cdot x + w_0
    \end{equation*}
    \begin{center}
      \scalebox{0.5}{
        \input{figures/chapter2/ovf_4_quadratic.pgf}
      }
    \end{center}
  \end{onlyenv}
  \begin{onlyenv}<6>
    3$^{rd}$ degree polynomial \(\rightarrow\) Good Fit!
    \begin{equation*}
      y = w_3\cdot x^3 + w_2\cdot x^2 + w_1\cdot x + w_0
    \end{equation*}
    \begin{center}
      \scalebox{0.5}{
        \input{figures/chapter2/ovf_5_3d.pgf}
      }
    \end{center}
  \end{onlyenv}
  \begin{onlyenv}<7>
    9$^{th}$ degree polynomial \(\rightarrow\) Overfitting!
    \begin{equation*}
      y = \sum_{i=0}^{9}w_i\cdot x^{i}
    \end{equation*}
    \begin{center}
      \scalebox{0.5}{
        \input{figures/chapter2/ovf_6_9d.pgf}
      }
    \end{center}
  \end{onlyenv}
\end{frame}

\begin{frame}
  \frametitle{Problem diagnosis}

  \begin{itemize}
  \item Model behavior is \emph{explained} by the shape of the function
  \item Overfitting, Underfitting are easily diagnosed
  \item If the input has multiple dimensions $D$?
    \begin{itemize}
    \item We often have tens or hundreds of features
    \item Images and signals: Several thousands of input dimensions
    \end{itemize}
  \end{itemize}
\end{frame}



\begin{frame}
  \frametitle{Bike Sharing Problem}

  \begin{itemize}
  \item Predict Bike rentals per hour in California
  \item We have 11 features
    \begin{itemize}
    \item e.g., month, hour, temperature, humidity, windspeed
    \end{itemize}
  \item We fit a Neural Network \(y = \hat{f}(\xb)\)
  \item How to make a plot like before?
    \begin{itemize}
    \item Feature Effect methods
    \end{itemize}
  \end{itemize}
\end{frame}


\section{Feature Effect Methods}

\begin{frame}
  \frametitle{Feature effect methods}
  \begin{itemize}
  \item High-dimensional input space \(\xb \in \mathbb{R}^D\)
    \begin{itemize}
    \item \(x_s \rightarrow \) feature of interest
    \item \(\Vx_c \rightarrow\) other features
    \end{itemize}
  \item How do we isolate the effect of \(x_s\)?
  \end{itemize}
\end{frame}


%\begin{frame}
%  \frametitle{Partial Dependence Plots (PDP)}
%  \begin{onlyenv}<1>
%    \begin{itemize}
%    \item Proposed by J. Friedman on 2001\footnote{J. Friedman. ``Greedy
%    function approximation: A gradient boosting machine.'' Annals of statistics
%    (2001): 1189-1232} and is the marginal \emph{effect} of a feature to the
%      model output:
%      \begin{equation*}
%        f_s(x_s) = E_{X_c}\left[\hat{f}(x_s, X_c)\right]
%      \end{equation*}
%    \item Computation:
%      \begin{equation*}
%        \hat{f}_s(x_s) = \frac{1}{n}\sum\limits_{i=1}^{n}\hat{f}(x_s, \Vx_c^{(i)})
%      \end{equation*}
%    \end{itemize}
%  \end{onlyenv}
%  \begin{onlyenv}<2>
%    \emph{Bike sharing Dataset:}
%    \begin{figure}
%      \includegraphics[width=\textwidth]{pdp-bike-1}
%      \caption{\footnotesize C. Molnar, IML book}
%    \end{figure}
%  \end{onlyenv}
%\end{frame}
%
%\begin{frame}
%  \frametitle{Issues with PDPs}
%  \begin{onlyenv}<1>
%    \begin{itemize}
%    \item The marginal distribution ignores correlated features!
%    \item To compute the effect of temperature $=33$ degrees it will (also) use an instance
%        with month = January
%    \end{itemize}
%    \begin{figure}
%      \includegraphics[width=.6\textwidth]{aleplot-motivation1-1}
%      \caption{\footnotesize C. Molnar, IML book}
%    \end{figure}
%  \end{onlyenv}
%\end{frame}
%
%
%\begin{frame}
%  \frametitle{Accumulated Local Effects (ALE)\footnote{D. Apley and
%    J. Zhu. ``Visualizing the effects of predictor variables in black box
%    supervised learning models.'' Journal of the Royal Statistical Society:
%    Series B (Statistical Methodology) 82.4 (2020): 1059-1086.}}
%
%  \begin{itemize}
%  \item Resolves problems that result from the feature correlation by computing
%    differences over a (small) window
%  \item Definition: \(f(x_s) = \int_{x_{min}}^{x_s}\mathbb{E}_{\Vx_c|z}[ \frac{\partial f}{\partial x_s}(z, \Vx_c)] \partial z\)
%  \end{itemize}
%\end{frame}
%
%\begin{frame}
%  \frametitle{ALE approximation}
%  Approximation: \(f(x_s) = \sum\limits_{k=1}^{k_x}
%  \underbrace{\frac{1}{|\mathcal{S}_k|} \sum_{i:\Vx^i \in \mathcal{S}_k}
%    \underbrace{[f(z_k, \Vx^i_c) - f(z_{k-1}, \Vx^i_c)]}_{\text{point
%        effect}}}_{\text{bin effect}} \)
%
%  \begin{figure}[ht]
%    \centering
%    \includegraphics[width=0.65\textwidth]{./figures/ale_bins_iml.png}
%    \caption{\footnotesize C. Molnar, IML book}
%  \end{figure}
%\end{frame}
%
%\begin{frame}
%  \frametitle{ALE plots - examples}
%  \begin{figure}
%    \includegraphics[width=1\textwidth]{ale-bike-1}
%    \caption{\footnotesize C. Molnar, IML book}
%  \end{figure}
%\end{frame}


\subsection{Interaction Indices}

\subsection{Feature Importance}

\section{Feature Interaction}
%\begin{frame}
  \frametitle{Feature Interaction - Motivation}
  \begin{itemize}
  \item Is Feature Effect a good approach?
    \begin{itemize}
    \item Interpretability? very good, easy intuition
    \item Fidelity? it depends..
    \end{itemize}
  \item Additive case: \(f(\xb) = f_1(x_1) + f_2(x_2)\)
    \begin{itemize}
    \item Generalized Additive Models
      \item X-by-design
    \end{itemize}

\item Non-additive case: \(f(\xb) = f_1(x_1) + f_2(x_2) + \underbrace{f_{12}(x_1, x_2)}_{interaction}\)
  \begin{itemize}
  \item how to distribute \(f_{12}(x_1, x_2)\) to \(x_1\) and \(x_2\)?
  \item Research question! Later: uncertainty could help
  \end{itemize}
  \item \(f\) is unknonw, so first, someone must inform about the interaction terms
  \item Feature Interaction Methods!
\end{itemize}
\end{frame}


\begin{frame}
  \frametitle{Problem Statement}
  When features interact with each other in a prediction model, the prediction cannot be expressed as the sum of the feature effects, because the effect of one feature depends on the value of the other feature. Aristotle’s predicate “The whole is greater than the sum of its parts” applies in the presence of interactions.\footnote{\href{https://christophm.github.io/interpretable-ml-book/interaction.html}{Interpretable Machine Learning book}}
\end{frame}


\begin{frame}
  \frametitle{H-statistic}
  \begin{itemize}
  \item Level of interaction between feature \(i\) and feature \(j\)
  \end{itemize}

  \[ \mathcal{H}^2_{jk} = \frac{\sum_{i=1}^n \left ( PD_{jk}(x^{(i)}_j, x^{(i)}_k) - PD_{j}(x^{(i)}_j) - PD_{k}(x^{(i)}_k)\right )^2 }{\sum_{i=1}^n PD^2_{jk}(x^{(i)}_j, x^{(i)}_k) } \]

  \begin{itemize}
  \item Level of interaction between feature \(i\) and all the other features
  \end{itemize}

    \[ \mathcal{H}^2_{j} = \frac{\sum_{i=1}^n \left ( f(x^{(i)}) - PD_{j}(x^{(i)}_j) - PD_{-j}(x^{(i)}_{-j})\right )^2 }{\sum_{i=1}^n f^2(x^{(i)}) } \]


\end{frame}


\begin{frame}
  \frametitle{H-statistic}
   \begin{figure}
     \includegraphics[width=\textwidth]{h_statistic}
     \caption{\footnotesize C. Molnar, IML book}
   \end{figure}

\end{frame}


\begin{frame}
  \frametitle{Other approaches}
  \begin{itemize}
  \item Greenwell's interaction index
    \begin{itemize}
    \item PDP-based method
      \item \href{https://arxiv.org/pdf/1805.04755.pdf}{A Simple and Effective Model-Based Variable Importance Measure}
    \end{itemize}
  \item SHAP interaction index
    \begin{itemize}
    \item SHAP-based method
    \item \href{https://arxiv.org/abs/1802.03888}{Consistent Individualized Feature Attribution for Tree Ensembles}
    \end{itemize}
  \end{itemize}
\end{frame}


\section{Heterogeneous effects}
%\begin{frame}
  \frametitle{Interaction implies heterogeneity}
  The interaction index measures the contribution of the interaction terms:

  \begin{itemize}
    \item \(f(x_1, x_2) = x_1^2 + \log(x_2) + \alpha x_1x_2^3\)
    \item \(\alpha = 0.1 \rightarrow\) low interaction index \(\rightarrow\) high fidelity
    \item \(\alpha = 100 \rightarrow\) high interaction index \(\rightarrow\) low fidelity
    \end{itemize}
  \noindent\makebox[\linewidth]{\rule{\paperwidth}{0.4pt}}
  But it does not say how the interaction terms influence the feature effect plots
\end{frame}

\begin{frame}
  \frametitle{Example}
  \begin{figure}
    \centering
    \includegraphics[width=1\textwidth]{pdp}
    \caption{PDP plot, taken from }
  \end{figure}
\end{frame}


\section{Feature Importance}
% \subsection{Feature Effect}

\begin{frame}
  \frametitle{Example}
  \begin{onlyenv}<1>
    Consider the following mapping $x \rightarrow y$
    \begin{center}
      \scalebox{0.5}{
        \input{figures/chapter2/ovf_1_reality.pgf}
      }
    \end{center}
  \end{onlyenv}
  \begin{onlyenv}<2>
    Process unknown \(\rightarrow\) we only have samples
    \begin{center}
      \scalebox{0.5}{
        \input{figures/chapter2/ovf_2_sampling.pgf}
      }
    \end{center}
  \end{onlyenv}
  \begin{onlyenv}<3>
    Our goal is to model the process using the available samples (regression)
    \vspace{1cm}\\
  \end{onlyenv}
  \begin{onlyenv}<4>
    Linear model \(\rightarrow\) Underfiting!
    \begin{equation*}
      y = w_1\cdot x + w_0
    \end{equation*}
    \begin{center}
      \scalebox{0.5}{
        \input{figures/chapter2/ovf_3_linear.pgf}
      }
    \end{center}
  \end{onlyenv}
  \begin{onlyenv}<5>
    2$^{nd}$ degree polynomial \(\rightarrow\) Decent Fit!
    \begin{equation*}
      y = w_2\cdot x^2 + w_1\cdot x + w_0
    \end{equation*}
    \begin{center}
      \scalebox{0.5}{
        \input{figures/chapter2/ovf_4_quadratic.pgf}
      }
    \end{center}
  \end{onlyenv}
  \begin{onlyenv}<6>
    3$^{rd}$ degree polynomial \(\rightarrow\) Good Fit!
    \begin{equation*}
      y = w_3\cdot x^3 + w_2\cdot x^2 + w_1\cdot x + w_0
    \end{equation*}
    \begin{center}
      \scalebox{0.5}{
        \input{figures/chapter2/ovf_5_3d.pgf}
      }
    \end{center}
  \end{onlyenv}
  \begin{onlyenv}<7>
    9$^{th}$ degree polynomial \(\rightarrow\) Overfitting!
    \begin{equation*}
      y = \sum_{i=0}^{9}w_i\cdot x^{i}
    \end{equation*}
    \begin{center}
      \scalebox{0.5}{
        \input{figures/chapter2/ovf_6_9d.pgf}
      }
    \end{center}
  \end{onlyenv}
\end{frame}

\begin{frame}
  \frametitle{Problem diagnosis}

  \begin{itemize}
  \item Model behavior is \emph{explained} by the shape of the function
  \item Overfitting, Underfitting are easily diagnosed
  \item If the input has multiple dimensions $D$?
    \begin{itemize}
    \item We often have tens or hundreds of features
    \item Images and signals: Several thousands of input dimensions
    \end{itemize}
  \end{itemize}
\end{frame}



\begin{frame}
  \frametitle{Bike Sharing Problem}

  \begin{itemize}
  \item Predict Bike rentals per hour in California
  \item We have 11 features
    \begin{itemize}
    \item e.g., month, hour, temperature, humidity, windspeed
    \end{itemize}
  \item We fit a Neural Network \(y = \hat{f}(\xb)\)
  \item How to make a plot like before?
    \begin{itemize}
    \item Feature Effect methods
    \end{itemize}
  \end{itemize}
\end{frame}


\section{Feature Effect Methods}

\begin{frame}
  \frametitle{Feature effect methods}
  \begin{itemize}
  \item High-dimensional input space \(\xb \in \mathbb{R}^D\)
    \begin{itemize}
    \item \(x_s \rightarrow \) feature of interest
    \item \(\Vx_c \rightarrow\) other features
    \end{itemize}
  \item How do we isolate the effect of \(x_s\)?
  \end{itemize}
\end{frame}


%\begin{frame}
%  \frametitle{Partial Dependence Plots (PDP)}
%  \begin{onlyenv}<1>
%    \begin{itemize}
%    \item Proposed by J. Friedman on 2001\footnote{J. Friedman. ``Greedy
%    function approximation: A gradient boosting machine.'' Annals of statistics
%    (2001): 1189-1232} and is the marginal \emph{effect} of a feature to the
%      model output:
%      \begin{equation*}
%        f_s(x_s) = E_{X_c}\left[\hat{f}(x_s, X_c)\right]
%      \end{equation*}
%    \item Computation:
%      \begin{equation*}
%        \hat{f}_s(x_s) = \frac{1}{n}\sum\limits_{i=1}^{n}\hat{f}(x_s, \Vx_c^{(i)})
%      \end{equation*}
%    \end{itemize}
%  \end{onlyenv}
%  \begin{onlyenv}<2>
%    \emph{Bike sharing Dataset:}
%    \begin{figure}
%      \includegraphics[width=\textwidth]{pdp-bike-1}
%      \caption{\footnotesize C. Molnar, IML book}
%    \end{figure}
%  \end{onlyenv}
%\end{frame}
%
%\begin{frame}
%  \frametitle{Issues with PDPs}
%  \begin{onlyenv}<1>
%    \begin{itemize}
%    \item The marginal distribution ignores correlated features!
%    \item To compute the effect of temperature $=33$ degrees it will (also) use an instance
%        with month = January
%    \end{itemize}
%    \begin{figure}
%      \includegraphics[width=.6\textwidth]{aleplot-motivation1-1}
%      \caption{\footnotesize C. Molnar, IML book}
%    \end{figure}
%  \end{onlyenv}
%\end{frame}
%
%
%\begin{frame}
%  \frametitle{Accumulated Local Effects (ALE)\footnote{D. Apley and
%    J. Zhu. ``Visualizing the effects of predictor variables in black box
%    supervised learning models.'' Journal of the Royal Statistical Society:
%    Series B (Statistical Methodology) 82.4 (2020): 1059-1086.}}
%
%  \begin{itemize}
%  \item Resolves problems that result from the feature correlation by computing
%    differences over a (small) window
%  \item Definition: \(f(x_s) = \int_{x_{min}}^{x_s}\mathbb{E}_{\Vx_c|z}[ \frac{\partial f}{\partial x_s}(z, \Vx_c)] \partial z\)
%  \end{itemize}
%\end{frame}
%
%\begin{frame}
%  \frametitle{ALE approximation}
%  Approximation: \(f(x_s) = \sum\limits_{k=1}^{k_x}
%  \underbrace{\frac{1}{|\mathcal{S}_k|} \sum_{i:\Vx^i \in \mathcal{S}_k}
%    \underbrace{[f(z_k, \Vx^i_c) - f(z_{k-1}, \Vx^i_c)]}_{\text{point
%        effect}}}_{\text{bin effect}} \)
%
%  \begin{figure}[ht]
%    \centering
%    \includegraphics[width=0.65\textwidth]{./figures/ale_bins_iml.png}
%    \caption{\footnotesize C. Molnar, IML book}
%  \end{figure}
%\end{frame}
%
%\begin{frame}
%  \frametitle{ALE plots - examples}
%  \begin{figure}
%    \includegraphics[width=1\textwidth]{ale-bike-1}
%    \caption{\footnotesize C. Molnar, IML book}
%  \end{figure}
%\end{frame}


\subsection{Interaction Indices}

\subsection{Feature Importance}

\section{Non-Tabular case}


\end{document}